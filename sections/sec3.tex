\documentclass[../main]{subfiles}
\begin{document}

\graphicspath{{../figures/}}

\section{検証実験}
本研究では,提案手法の有効性を検証するために,マイクを搭載した台車を移動ロボットに見立て,横2.3m,縦3mの経路を走行させ,異常音源の座標を推定する実験を行った.
正常音のデータは,ギアボックスの運転音を収録したものを用い,異常音のデータは,異物をかますことで発生した音を収録したものを用いた.
正常音のデータと異常音のデータの取得回数はそれぞれ4セットと1セットであり,正常音のデータのうち3セットを学習データ,1セットを検証データとして用いた.
以下に,移動ロボットの経路とギアボックスの音源の配置を示す.
異常音源の正確な座標は,\reffig{route}に示す経路上の座標(0.5, -0.3)とした.
実験に用いたマイクのサンプリング周波数は44.1kHz,前処理に用いた短時間フーリエ変換のウィンドウサイズは65536,スライド幅は1024,メルフィルタバンクの数は128とした.
\begin{figure}[tb]
  \centering
  \includegraphics[keepaspectratio, width=0.8\linewidth]{route.pdf}
  \caption{移動ロボットの経路}
  \labfig{route}
\end{figure}

\end{document}
