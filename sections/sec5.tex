\documentclass[../main]{subfiles}
\begin{document}

\graphicspath{{../figures}}

\section{結論}
本研究では異常音の教師データを用いることなく,異常音源の座標を推定することに成功した.
また,ニューラルネットワークを用いて正常音をマッピングし,ニューラルネットワークの出力との差分から異常音源の座標を推定する手法を提案し,
提案手法の有効性を実験を通して示した.
しかしながら,本研究にて提案した手法では,ロボットの経路が直線である場合には異常音源の座標を推定することができないという課題が存在する.


今後の展望としてはマイクアレイ,もしくは指向性のあるマイクを用いることで,異常音源の座標を経路が直線である場合でも推定できるようにすることが挙げられる.
また,異常音源の座標を推定する際に,音のエネルギーが距離の二乗に反比例して減衰するという性質を前提として用いているが,実際の環境下では,音の反射や吸収により音のエネルギーの減衰の仕方が異なることが考えられ,
この課題に対しても異常音源の座標を高精度で推定できるよう手法を改善することが今後の課題として挙げられる.
\end{document}
