\documentclass[../main]{subfiles}
\begin{document}

\graphicspath{{../figures/}}

\section{序論}
本文は2段組みとする.
改行を入れて読みやすく書く.

引用例は\cite{Schlick1994}\cite{Chikushi2020}このようになる.

石油精製プラント内では,図1に示すような危機の老朽化や予期せぬ不調を原因とする
ポンプにおけるベアリングの異常など様々な異常が存在し,
ベアリングやポンプなどの回転機器が対象の場合は,それらから発せられる音を利用して
異常か正常化の判断が行われる.

従来それらの音による異常の検知は点検員による巡回点検によって行われているが,
人手による巡回点検には,高齢化による熟練点検員の不足や,非熟練点検員による見落としが発生してしまうといった問題があげられる.

それらの理由から石油精製プラント内における音響点検を自動化することが求められている.


プラント内における音響点検の自動化には,主に2つのアプローチが考えられ,固定点マイクを用いる手法と,
移動ロボットにマイクを搭載する手法がある.

しかしながら,石油精製プラントは点検対象の面積が非常に広大であり,固定点マイクを用いた手法では,必要となるマイク数が膨大になってしまう点,
また,石油精製プラントでは,狭い区間に多くの音を発する危機が混在しているため,異常音の発生源を特定することが困難である点が課題となっている.

これらの理由から,本研究では移動ロボットにマイクを搭載し,プラント内の音響点検を自動化することを目指す.

石油精製プラント内には,ポンプ,配管,スチームトラップなど様々な機器が存在し,それらの機器からは様々な音が発せられる.
そのため,場所によって収音される音の特性が異なる.このような石油精製プラント内における異常音検知手法として,Fujita~\cite{fujita2022}
は,移動ロボットの経路を区間に分割し,区間ごとに異なる異常検知モデルを学習し,運用時は自己位置に対応する異常検知モデルを選択する手法を提案している.
しかしながら,この研究では,分割した区間の解像度でしか異常音源の座標を推定することができないという課題が存在する.


そのため,本研究では教師データが不要で異常音源の座標を推定可能な,移動ロボットによる異常音検知手法を目指す.


\end{document}
